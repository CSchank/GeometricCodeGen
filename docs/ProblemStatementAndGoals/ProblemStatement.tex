\documentclass{article}

\usepackage{tabularx}
\usepackage{booktabs}
\usepackage[compress, numbers]{natbib} % Bibliography formatting

\title{Problem Statement and Goals\\\progname}

\author{\authname}

\date{}

\input{../Comments}
%% Common Parts

\newcommand{\progname}{Drasil Matrix, Vector and Tensor Extension} % PUT YOUR PROGRAM NAME HERE
\newcommand{\authname}{Christopher W. Schankula} % AUTHOR NAMES                  

\usepackage{hyperref}
    \hypersetup{colorlinks=true, linkcolor=blue, citecolor=blue, filecolor=blue,
                urlcolor=blue, unicode=false}
    \urlstyle{same}
                                


\begin{document}

\maketitle

\begin{table}[hp]
\caption{Revision History} \label{TblRevisionHistory}
\begin{tabularx}{\textwidth}{llX}
\toprule
\textbf{Date} & \textbf{Developer(s)} & \textbf{Change}\\
\midrule
January 17th, 2025 & Christopher Schankula & Initial version of problem statement\\
\bottomrule
\end{tabularx}
\end{table}

\section{Problem Statement}

This section describes the problem that will be solved in this project. We start by
presenting background information related to matrices, vectors and tensors. We then
provide a high-level description of the problem, followed by an overview of the
stakeholders and environment in which the resulting software will run.

\subsection{Background}

This section describes background on matrices, vectors, tensors, and the Drasil
project.

\subsubsection{Matrices, Vectors, and Tensors}

A matrix is a rectangular table of numbers, symbols or expressions which
are used to represent some mathematical object or concept. They are used in
linear algebra, physics, and number theory, among other 
fields~\cite{Wikipedia_Matrix_2025}. Meanwhile, a vector is a ``quantities that 
cannot be expressed by a single number (a scalar)''~\cite{Wikipedia_Vectors_2024}.
A vector can be seen as a sort of one-dimensional matrix. Similarly to matrices,
vectors are used in a wide variety of scientific fields.

A tensor is a type of mathematical objects that describe a relationship
between matrices, vectors, or other tensors~\cite{Wikipedia_Tensors_2025}. 
The relationship is a multilinear map, meaning it is a function of several 
linear inputs~\cite{Wikipedia_Multilinear_2024}. Tensors can be used to solve
important physics problems, including in mechanics, electrodynamics, and general
relativity~\cite{Wikipedia_Tensors_2025}.

\subsection{Drasil Project}

\wss{You should check your problem statement with the
\href{https://github.com/smiths/capTemplate/blob/main/docs/Checklists/ProbState-Checklist.pdf}
{problem statement checklist}.} 

\wss{You can change the section headings, as long as you include the required
information.}

\subsection{Problem}
Currently, vectors and matrices in Drasil are implemented in a way that does not 
promote the correctness of algorithms using them. For calculations on matrices,
vectors and tensors to be correct, it is necessary to ensure certain properties
about them, including the sizes of operands. Furthermore, current implementations
of matrices and vectors in Drasil support only fixed fixed sizes. 

The problem being solved in this project will be to allow the specification of
tensors, matrices, and vectors and document and code  generation associated with them.

\subsection{Inputs and Outputs}

The inputs to the system will be a representation of operations using tensors
(matrices and vectors will be built on top of tensors). The outputs from the
system will be documents describing the operations specified in the inputs and
code in several supported languages to compute the operations described.

% \wss{Characterize the problem in terms of ``high level'' inputs and outputs.  
% Use abstraction so that you can avoid details.}

\subsection{Stakeholders}

\subsection{Environment}

\wss{Hardware and software environment}

\section{Goals}

\section{Stretch Goals}

\section{Challenge Level and Extras}

\wss{State your expected challenge level (advanced, general or basic).  The
challenge can come through the required domain knowledge, the implementation or
something else.  Usually the greater the novelty of a project the greater its
challenge level.  You should include your rationale for the selected level.
Approval of the level will be part of the discussion with the instructor for
approving the project.  The challenge level, with the approval (or request) of
the instructor, can be modified over the course of the term.}

\wss{Teams may wish to include extras as either potential bonus grades, or to
make up for a less advanced challenge level.  Potential extras include usability
testing, code walkthroughs, user documentation, formal proof, GenderMag
personas, Design Thinking, etc.  Normally the maximum number of extras will be
two.  Approval of the extras will be part of the discussion with the instructor
for approving the project.  The extras, with the approval (or request) of the
instructor, can be modified over the course of the term.}

\newpage{}

\section{References}
\bibliographystyle{abbrvnat}
\bibliography{Bib}

\end{document}
