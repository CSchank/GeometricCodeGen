\documentclass{article}

\usepackage{tabularx}
\usepackage{booktabs}
\usepackage[compress, numbers]{natbib} % Bibliography formatting
\usepackage{enumitem}

\title{Problem Statement and Goals\\\progname}

\author{\authname}

\date{}

\input{../Comments}
%% Common Parts

\newcommand{\progname}{Drasil Matrix, Vector and Tensor Extension} % PUT YOUR PROGRAM NAME HERE
\newcommand{\authname}{Christopher W. Schankula} % AUTHOR NAMES                  

\usepackage{hyperref}
    \hypersetup{colorlinks=true, linkcolor=blue, citecolor=blue, filecolor=blue,
                urlcolor=blue, unicode=false}
    \urlstyle{same}
                                


\begin{document}

\maketitle

\begin{table}[hp]
\caption{Revision History} \label{TblRevisionHistory}
\begin{tabularx}{\textwidth}{llX}
\toprule
\textbf{Date} & \textbf{Developer(s)} & \textbf{Change}\\
\midrule
January 19th, 2025 & Christopher Schankula & Initial version of problem statement\\
\bottomrule
\end{tabularx}
\end{table}

\section{Problem Statement}

This section describes the problem that will be solved in this project. We start by
presenting background information related to matrices, vectors and tensors. We then
provide a high-level description of the problem, followed by an overview of the
stakeholders and environment in which the resulting software will run.

\subsection{Background}

This section describes background on matrices, vectors, tensors, and the Drasil
project.

\subsubsection{Matrices, Vectors, and Tensors}

A matrix is a rectangular table of numbers, symbols or expressions which
are used to represent some mathematical object or concept. They are used in
linear algebra, physics, and number theory, among other 
fields~\cite{Wikipedia_Matrix_2025}. Meanwhile, a vector is a ``quantities that 
cannot be expressed by a single number (a scalar)''~\cite{Wikipedia_Vectors_2024}.
A vector can be seen as a sort of one-dimensional matrix. Similarly to matrices,
vectors are used in a wide variety of scientific fields.

A tensor is a type of mathematical objects that describe a relationship
between matrices, vectors, or other tensors~\cite{Wikipedia_Tensors_2025}. 
The relationship is a multilinear map, meaning it is a function of several 
linear inputs~\cite{Wikipedia_Multilinear_2024}. Tensors can be used to solve
important physics problems, including in mechanics, electrodynamics, and general
relativity~\cite{Wikipedia_Tensors_2025}.

\subsection{Drasil Project}

Drasil is a system that takes advantage of the inherent duplication in knowledge in a
software engineering project (e.g.~code, specification, testing, etc.). By capturing
this knowledge once and generating these artifacts, they can be kept up-to-date and
lead to more robust software~\cite{Carette_Drasil_2021}.

% \wss{You should check your problem statement with the
% \href{https://github.com/smiths/capTemplate/blob/main/docs/Checklists/ProbState-Checklist.pdf}
% {problem statement checklist}.} 

% \wss{You can change the section headings, as long as you include the required
% information.}

\subsection{Problem}
Currently, vectors and matrices in Drasil are implemented in a way that does not 
promote the correctness of algorithms using them. For calculations on matrices,
vectors and tensors to be correct, it is necessary to ensure certain properties
about them, including the sizes of operands. Furthermore, current implementations
of matrices and vectors in Drasil support only fixed sizes, and tensors are not
yet implemented at all.

The problem being solved in this project will be to allow the specification of
tensors, matrices, and vectors in the Drasil system and document and code generation 
associated with them.

\subsection{Inputs and Outputs}

The inputs to the system will be a representation of operations using tensors
(matrices and vectors will be built on top of tensors). The outputs from the
system will be documents describing the operations specified in the inputs and
code in several supported languages to compute the operations described.

% \wss{Characterize the problem in terms of ``high level'' inputs and outputs.  
% Use abstraction so that you can avoid details.}

\subsection{Stakeholders}
This project has the following stakeholders:

\begin{itemize}
\item Dr.~Carette and Dr.~Smith: Project supervisors
\item Researchers interested in software engineering
\item Software engineers looking to develop safe and correct software
\end{itemize}

\subsection{Environment}

The software environment of this project is the Haskell language since Drasil is
written in Haskell. Thus, it is a constraint that this project will be written
in Haskell and will extend the existing Drasil project and software environment
in which it is defined.

There is no particular hardware environment for this project. Any hardware able to
support compiling Haskell code is considered in-scope.

% \wss{Hardware and software environment}

\section{Goals}

This project consists of five main goals and two stretch goals. The main goals define
what the project must achieve to be considered a success. Stretch goals represent
additional future goals worth working towards if time/resources allow (and whose
considerations are worth keeping in mind as the main system is built), but which 
are considered out of scope for the current project. The main goals are as follows:

\begin{enumerate}[label={G\arabic*.}]
    \item Expand Drasil's expression language with new primitives for generalized tensors.
    \begin{enumerate}
    \item[G1.1] Define smart constructors (built on top of the generalized tensors) for 
          the simple cases of vectors and matrices.
    \end{enumerate}
    \item Expand Drasil's type system to cover tensors, vectors and matrices.
    \item Enable documentation and code generation using these new primitives.
    \item Include enough information in the types to support at least code runtime 
          checks of sizes on tensor/vector/matrix operations.
    \item Demonstrate the use of this system to generate documents and code for basic 
          scientific computing tasks using vectors, matrices and/or tensors.
    \begin{itemize}
    \item[G5.1.] Ensure that existing Drasil examples continue to work despite these
          new changes.
    \end{itemize}
\end{enumerate}

\section{Stretch Goals}

The following are stretch goals considered out of scope for the current project but
are goals worth working towards in the future:

\begin{enumerate}[label={SG\arabic*.}]
\item Demonstrate the use of this system for basic image processing tasks (e.g. 
mean/median filters).
\item Enable type safety of operations at specification-time.
\end{enumerate}

\section{Challenge Level and Extras}

The challenge level for this project is \textbf{Research}. Exact deliverables will be
determined in conjunction with the instructor. The extra being proposed is to write
a research paper describing the work and showing the examples. The paper is to be
published on arXiv.org at a minimum, with hopes to publish in a journal and/or 
conference.

% \wss{State your expected challenge level (advanced, general or basic).  The
% challenge can come through the required domain knowledge, the implementation or
% something else.  Usually the greater the novelty of a project the greater its
% challenge level.  You should include your rationale for the selected level.
% Approval of the level will be part of the discussion with the instructor for
% approving the project.  The challenge level, with the approval (or request) of
% the instructor, can be modified over the course of the term.}

% \wss{Teams may wish to include extras as either potential bonus grades, or to
% make up for a less advanced challenge level.  Potential extras include usability
% testing, code walkthroughs, user documentation, formal proof, GenderMag
% personas, Design Thinking, etc.  Normally the maximum number of extras will be
% two.  Approval of the extras will be part of the discussion with the instructor
% for approving the project.  The extras, with the approval (or request) of the
% instructor, can be modified over the course of the term.}

\newpage{}

\bibliographystyle{abbrvnat}
\bibliography{Bib}

\end{document}
