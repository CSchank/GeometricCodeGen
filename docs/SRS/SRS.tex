%----------------------------------------------------------------------------------------
%    PACKAGES AND THEMES
%----------------------------------------------------------------------------------------

\documentclass[aspectratio=169,xcolor=dvipsnames]{beamer}
\setbeamertemplate{caption}[numbered]
\usetheme{SimpleDarkBlue}

\usepackage{hyperref}
\usepackage{graphicx} % Allows including images
\usepackage{booktabs} % Allows the use of \toprule, \midrule and \bottomrule in tables
\usepackage{cmbright}
\usepackage{tabularx}
\usepackage{subcaption}
\usepackage[round]{natbib}

\fontencoding{OT1}\fontfamily{cmbr}\selectfont %to load ot1cmbr.fd
\DeclareFontShape{OT1}{cmbr}{bx}{n}{% change bx definition
<->cmbrbx10%
}{}
\normalfont % back to normalfont

%----------------------------------------------------------------------------------------
%    TITLE PAGE
%----------------------------------------------------------------------------------------

\title{Drasil Geometric Algebra Extension}
\subtitle{Overview of Requirements}

\author{Christopher Schankula}

\institute
{
    Department of Computing \& Software \\
    McMaster University % Your institution for the title page
}
\date{\today} % Date, can be changed to a custom date

%----------------------------------------------------------------------------------------
%    PRESENTATION SLIDES
%----------------------------------------------------------------------------------------

\begin{document}

\begin{frame}
    % Print the title page as the first slide
    \titlepage
\end{frame}

\begin{frame}{Overview}
    % Throughout your presentation, if you choose to use \section{} and \subsection{} commands, these will automatically be printed on this slide as an overview of your presentation
    \tableofcontents

    The purpose of this slide deck is to define the objects and operations that will
    be implemented in the project. We will begin with a discussion about vectors and
    matrices, then generalize with geometric algebra. Next, we will redefine the
    operations using geometric algebra. Finally, we will finish with a discussion of
    some high-level requirements for the project.
\end{frame}

\begin{frame}{Revision History}
      \begin{tabularx}{\textwidth}{p{4cm}p{2cm}X}
            \toprule {\bf Date} & {\bf Version} & {\bf Notes}\\
            \midrule
            January 23rd, 2025 & 1.0 & First draft of document version for presentation\\
            February 7th, 2025 & 1.1 & First draft of slide version of document\\
            \bottomrule
      \end{tabularx}
\end{frame}

%------------------------------------------------
\section{Einstein Summation Notation}
%------------------------------------------------

\begin{frame}{Einstein Summation Notation}
    \begin{itemize}
      \item Used to describe repeated summations and multiplications in a compact notation.
      \item They behave according to \textbf{four rules}~\cite{Khan2023}.
            \begin{enumerate}
            \item Any twice-repeated index in a single term is summed over.
            \item A twice-repeated index is called a dummy index; a once-repeated 
                  index is called a free index.
            \item No index may occur 3 or more times in a given term.
            \item In an equation with Einstein notation, the free indices on both sides must
                  match.
            \end{enumerate}
    \end{itemize}
\end{frame}

\begin{frame}{Einstein Summation Notation Rules}
      Here are the rules with some more explanations~\cite{Khan2023}:
      \begin{enumerate}
      \item<only@1> \textit{Any twice-repeated index in a single term is summed over.}\\
            For example, $a_{ij}b_i$ represents the term $\sum_{i=1}^j a_{ij}b_i$.
      \item<only@1> \textit{A twice-repeated index is called a dummy index; a once-repeated 
            index is called a free index.}\\
            In the example above, $i$ is a dummy index \textemdash~ it can be renamed
            however you would like. However, $j$ is a free index and has restrictions on naming.
      \item<only@1> \textit{No index may occur 3 or more times in a given term.}\\
            For example, $a_{ii}b_i$ is not legal.
      \item<only@2>{\textit{In an equation with Einstein notation, the free indices on both sides must
            match.}\\

            \noindent Some examples of correctly-formed equations:

            \begin{itemize}
            \item $x_i = a_{i,j}b_i$ is valid because $i$ is free on both the LHS and RHS
            \item $a_i = A_{k,i}B_{k,j}x_j + C_{i,k}u_k$ is valid because $i$ is a free variable on
                  the LHS, and in every term it is the free variable on the RHS.
            \end{itemize}

            \noindent Some examples of incorrectly-formed equations:

            \begin{itemize}
            \item $x_i = A_{j,i}$ is invalid because $i$ is the only free variable on the LHS, but
                  $i$ and $j$ are both free on the RHS.
            \item $x_j = A{i,k}u_k$ is invalid because $j$ is free on the LHS, but $i$ is free on 
                  the RHS.
            \item $x_i = A_{i,k}u_k + c_j$ is invalid because $i$ is free on the LHS, but on the 
                  RHS, one term has $i$ free while the other term has $j$ free.
            \end{itemize}}

      \end{enumerate}
      We will use this convention from now on. Note that this convention also allows us
      to drop any mention of ``$\forall i . 1 \leq i \leq n$'' since this is implied, so
      we will not write these to keep things clean.
  \end{frame}

%------------------------------------------------



%------------------------------------------------
\section{Vectors}
%------------------------------------------------

\begin{frame}{Vectors}
      \begin{itemize}
      \item A vector is a quantity having a magnitude and a direction~\cite{Wikipedia_Vectors_2024} (geometric interpretation).
      \item Used in mathematics and physics problems
      \item Vectors support two main operations: vector addition and scalar multiplication
      \item Notation
            \begin{itemize}
                  \item As a convention, we will denote vectors with lowercase boldface
                        font, such as $\mathbf{v}$.
                  \item The type of a vector can be written as $\mathbb{R}^n$, meaning
                        a vector of real numbers, with length $n$
            \end{itemize}
      \end{itemize}
\end{frame}

\begin{frame}{Vector Operations}
      \only<1>{\noindent\textbf{Vector Addition}

      \noindent Two vectors with the same size can be added together. Adding them adds their
      components together. That is,
      
      $$\mathbf{a} + \mathbf{b} = \mathbf{a}_i + \mathbf{b}_i$$
      
      \noindent\textbf{Scalar Multiplication}

      The scalar multiplication of a vector is equivalent to multiplying the magnitude
      by that number. Equivalently, it is the same as multiplying each component by the
      number. That is, a vector $\mathbf{a}$ scaled by the real number $r$ is denoted
      $r\mathbf{a}$ and is defined as:
      
      $$(r\mathbf{a})_i = r\mathbf{a}_i$$
      
      \noindent\textbf{Vector Subtraction}

      Vector subtraction can be defined in terms of addition and scalar multiplication:
      
      $$\mathbf{a} - \mathbf{b} = \mathbf{a}_i + (-1)\mathbf{b}_i$$}
      
      \only<2>{\noindent \textbf{Dot Product}
      
      The \textit{dot product} of two vectors each of size $n$ is represented by
      $$\mathbf{a} \cdot \mathbf{b} = \mathbf{a}_i\mathbf{b}_i$$.
      Some notes:
      \begin{itemize}
            \item $a \cdot a = \|a\|^2$, where $\|a\|$ is the length of $a$
            \item $a \cdot b = 0$ if $a$ and $b$ are perpendicular
      \end{itemize}
      
      Notice that it takes two vectors and outputs a real number.

      \noindent \textbf{Cross Product}
      
      The \textit{cross product} $\mathbf{c}$ of 3-dimensional vectors $\mathbf{a}$ and
      $\mathbf{b}$ is denoted as $\mathbf{c} = \mathbf{a} \times 
      \mathbf{b}$, and is defined by Einstein notation as~\cite{ContinuumMechanics2012}:
      
      $$\mathbf{c}_i = \epsilon_{i,j,k}\mathbf{a}_j\mathbf{b}_k\text{, where }\epsilon_{1,2,3}=
      \epsilon_{2,3,1}=\epsilon_{3,1,2} = 1\text{ and }\epsilon_{3,2,1} = \epsilon_{2,1,3}
      = \epsilon_{1,3,2} = -1$$}
\end{frame}

%------------------------------------------------


\begin{frame}{Sidenote: Linear Transformations}
      \begin{itemize}
            \item Let $T : \mathbb{R}^m \to \mathbb{R}^n$ be a function on vectors.
            \item $T$ is said to be a linear transformation if, for all scalars $k$ and $p$
                  and vectors $\mathbf{a}$ and $\mathbf{b}$, the following linearity
                  property holds:
                  $$T(k\mathbf{a} + p\mathbf{b}) = kT(\mathbf{a}) + pT(\mathbf{b})$$
                  In other words, if we scale the input vectors and add them together before
                  passing the result into the function, the result is the same as
                  scaling and adding after passing the vectors into the function~\cite{kuttler_first_2008}.
      \end{itemize}
\end{frame}
%------------------------------------------------


\begin{frame}{Basis Vectors}
      \begin{itemize}
            \item \textbf{Basis vector}: A set $B$ of vectors in a vector space $V$ such that every 
            vector in $V$ can be written as a unique linear combination of the vectors 
            in $B$~\cite{Wikipedia_Basis_2025}.
            \item A basis must be:
            \begin{itemize}
                  \item Linearly independent: for every finite subset $\{\mathbf{v}_1,\ldots,\mathbf{v}_n\}$
                        of $B$, if $c_1\mathbf{v}_1+\ldots c_n\mathbf{v}_n = \mathbf{0}$ then
                        $c_1=\ldots= c_n = 0$. In other words, no basis vector should be able to
                        be written as a linear combination of other basis vectors.
                  \item A spanning set: for every vector $\mathbf{v}$ in $V$, one can
                        write $\mathbf{v}$ as $\mathbf{v} = a_1\mathbf{v}_1+\ldots+a_n\mathbf{v}_n$.
                        In other words, the basis must be able to construct all vectors in the set.
                  \item For a given vector space $V$, every basis set $B$ will be of the same size.
                        This is called the \textit{dimension} of the vector space.
            \end{itemize}
      \end{itemize}
\end{frame}

% %------------------------------------------------

% \begin{frame}{Blocks of Highlighted Text}
%     In this slide, some important text will be \alert{highlighted} because it's important. Please, don't abuse it.

%     \begin{block}{Block}
%         Sample text
%     \end{block}

%     \begin{alertblock}{Alertblock}
%         Sample text in red box
%     \end{alertblock}

%     \begin{examples}
%         Sample text in green box. The title of the block is ``Examples".
%     \end{examples}
% \end{frame}

% %------------------------------------------------

% \begin{frame}{Multiple Columns}
%     \begin{columns}[c] % The "c" option specifies centered vertical alignment while the "t" option is used for top vertical alignment

%         \column{.45\textwidth} % Left column and width
%         \textbf{Heading}
%         \begin{enumerate}
%             \item Statement
%             \item Explanation
%             \item Example
%         \end{enumerate}

%         \column{.45\textwidth} % Right column and width
%         Lorem ipsum dolor sit amet, consectetur adipiscing elit. Integer lectus nisl, ultricies in feugiat rutrum, porttitor sit amet augue. Aliquam ut tortor mauris. Sed volutpat ante purus, quis accumsan dolor.

%     \end{columns}
% \end{frame}

%------------------------------------------------
\section{Matrices}
%------------------------------------------------

\begin{frame}{Matrices}

\begin{itemize}
      \item A \textit{matrix} is a rectangular table of numbers, symbols, or expressions
      with entries arranged in rows and columns~\cite{Wikipedia_Matrix_2025}. For
      example, the following is a 2-by-3 matrix: 
      
      $$\mathbf{A} = 
      \begin{bmatrix}
        a_{1,1} & a_{1,2} & a_{1,3} \\
        a_{2,1} & a_{2,2} & a_{2,3}
      \end{bmatrix}$$
      \item Matrices are also widely used in mathematics and physics.
      \item Notation
            \begin{itemize}
                  \item By convention, we will denote matrices as boldface
                        uppercase letters, e.g. $\mathbf{A}$
                  \item Sometimes, you will see the notation $\mathbb{R}^{m\times n}$
                        representing an $m \times n$ matrix of real numbers.
            \end{itemize}
\end{itemize}

\end{frame}

%------------------------------------------------

\begin{frame}{Matrix Operations}

\only<1>{\noindent\textbf{Addition}

\noindent The addition of two $m \times n$ matrices is calculated as:

$$(\mathbf A + \mathbf B)_{i,j} = \mathbf{A}_{i,j} + \mathbf{B}_{i,j}$$

\medskip
\noindent\textbf{Scalar Multiplication}
\noindent The scalar multiplication of a real number $c$ and a matrix $\mathbf{A}$ 
is denoted as $c\mathbf{A}$ and is computed as:

$$(c\mathbf A)_{i,j} = c\mathbf{A}_{i,j}$$

\medskip
\noindent\textbf{Subtraction}

\noindent The subtraction of two $m \times n$ matrices is denoted as $\mathbf{A} - \mathbf{B}$,
and is the addition of the scalar multiplication by -1:

$$(\mathbf A - \mathbf B)_{i,j} = \mathbf{A}_{i,j} + (-1)\mathbf{B}_{i,j}$$}


\only<2>{\medskip
\noindent\textbf{Transposition}

\noindent The \textit{transpose} of an $m\times n$ matrix is the $n\times m$ matrix $\mathbf{A}^T$
formed by turning rows into columns and vice versa:

$$(\mathbf A^T)_{i,j} = \mathbf A_{j,i}$$

\medskip
\noindent\textbf{Matrix Multiplication}

\noindent The \textit{matrix multiplication} of matrices $\mathbf A$ and $\mathbf B$, is 
defined when matrix $\mathbf A$ is of size $m \times p$ and matrix $\mathbf B$ is of 
size $p \times n$. Then, the resulting matrix $\mathbf A\mathbf B$ is a matrix of size 
$m \times n$, such that

$$(\mathbf A\mathbf B)_{i,j} = a_{i,r}b_{r,j}$$}

\end{frame}

%------------------------------------------------
\begin{frame}{Matrix Row Operations}

There are three kinds of row operations:

\begin{enumerate}
      \item<1-2> Row addition: adding one row to another
            \only<2>{This can be done by using a multiplying by a matrix that looks like
            an identity matrix, but with a non-zero entry in one of the non-diagonal
            spaces:
            $$\mathbf{A} = 
            \begin{bmatrix}
                  1 & 0 \\
                  2 & 1
                \end{bmatrix} 
            \begin{bmatrix}
              a & b \\
              c & b
            \end{bmatrix}
            = 
            \begin{bmatrix}
                  a & b \\
                  c + 2a & d + 2b
                \end{bmatrix} 
            $$
            }
      \item<1,3> Row multiplication: multiplying all entries in a row by a non-zero constant
            \only<3>{This can be done by an identity matrix with a non-one in the diagonal:
            $$\mathbf{A} = 
            \begin{bmatrix}
                  2 & 0 \\
                  0 & 1
                \end{bmatrix} 
            \begin{bmatrix}
              a & b \\
              c & b
            \end{bmatrix}
            = 
            \begin{bmatrix}
                  2a & 2b \\
                  c & d
                \end{bmatrix} 
            $$
            }
      \item<1,4> Row switching: interchanging two rows of a matrix
            \only<4>{This can be done by swapping the positions of the ones in the
            identity matrix:
            $$\mathbf{A} = 
            \begin{bmatrix}
                  0 & 1 \\
                  1 & 0
                \end{bmatrix} 
            \begin{bmatrix}
              a & b \\
              c & b
            \end{bmatrix}
            = 
            \begin{bmatrix}
                  d & c \\
                  a & b
                \end{bmatrix} 
            $$
            }
\end{enumerate}

\end{frame}

%------------------------------------------------
\begin{frame}{Submatrix}

Taking a submatrix involves taking only a portion of the rows and/or columns in
the matrix. To delete the rows or columns you want, you multiply by the identity
matrix with that row or column deleted.
For example, to extract the bottom-right $2 \times 2$ matrix, we must
first select the rows, then the columns that we want~\cite{mathse1322946}:
$$\mathbf{A} = 
\begin{bmatrix}
      0 & 1 & 0 \\
      0 & 0 & 1
\end{bmatrix} 
\begin{bmatrix}
  a & b & c \\
  d & e & f \\
  g & h & i
\end{bmatrix}
\begin{bmatrix}
      0 & 0 \\
      1 & 0 \\
      0 & 1 \\
\end{bmatrix} 
= 
\begin{bmatrix}
      e & f \\
      h & i
    \end{bmatrix} 
$$

\end{frame}

% %------------------------------------------------

% \begin{frame}{Theorem}
%     \begin{theorem}[Mass--energy equivalence]
%         $E = mc^2$
%     \end{theorem}
% \end{frame}

% %------------------------------------------------

% \begin{frame}{Figure}
%     Uncomment the code on this slide to include your own image from the same directory as the template .TeX file.
%     %\begin{figure}
%     %\includegraphics[width=0.8\linewidth]{test}
%     %\end{figure}
% \end{frame}

%------------------------------------------------
\section{Geometric Algebra}
%------------------------------------------------


%------------------------------------------------

\begin{frame}{Geometric Algebra}
\begin{itemize}
      \item \textit{Geometric algebra}, also known as \textit{Clifford algebra},
            is an abstraction of vectors and matrices.
      \item It allows us to think about more abstract objects, beyond just
            scalars and vectors, which are called \textit{bivectors}, 
            \textit{trivectors}, etc.
      \item Clifford Algebra defines many rules and operations, many of which have
            geometric interpretations.
      \item We will first define these, then continue to redefine vector and matrix
            operations using them.
      \item Notation
            \begin{itemize}
                  \item We will use $s$ to represent scalars, $V$ to represent vectors,
                        and $B$ to represent bivectors.
                  \item $A$, $B$, and $C$ will be used to represent any geometric algebra 
                        object (also known as clifs)
            \end{itemize}
      \item Unless otherwise indicated, this content comes from~\cite{henle2025clifford}.
            It is a great resource if you'd like to learn more.
\end{itemize}

\end{frame}

%------------------------------------------------



%------------------------------------------------

\begin{frame}{Visalizing Scalars, Vectors, Bivectors, $\ldots$}
\begin{itemize}
\item Most people know about how to visualize scalars (a point in space) and
      vectors (an arrow in space), but how do we visualize higher-dimensional objects?
\item The visualization is shown in Figure~\ref{Fig:ObjViz}
\item A bivector is a patch of flat surface
\item A trivector is a piece of three-dimensional space, with a volume and orientation.
\end{itemize}

\begin{figure}
      \includegraphics[width=0.7\textwidth]{Figs/blades.png}
      \caption{Visualization of geometric algebraic objects}\label{Fig:ObjViz}
\end{figure}

\end{frame}

%------------------------------------------------


\begin{frame}{Grades}
\begin{itemize}
\item Each of these (scalar, vector, bivector, $\ldots$) has a \textit{grade}, which
      corresponds to the number of dimensions involved. We have summarized these
      in Table~\ref{Tab:Grades}.
\item We call these objects (regardless of their grade) \textit{clifs}.
\end{itemize}

\begin{table}
\caption{Grades of Geometric Objects}\label{Tab:Grades}
\begin{tabularx}{\textwidth}{p{2cm}p{3cm}p{5cm}l}
      \toprule {\bf Object} & {\bf Visualized as} & {\bf Geometric Extent} & \bf Grade\\
      \midrule
      Scalar & Point & No geometric extent & 0 \\
      Vector & Line segment & Extent in 1 direction & 1 \\
      Bivector & Patch of surface & Extent in 2 directions & 2 \\
      Trivector & Piece of shape & Extent in 3 directions & 3 \\
      Etc. & & &\\
      \bottomrule
\end{tabularx}
\end{table}

\end{frame}

%------------------------------------------------

\begin{frame}{Visualization of Scaling Objects}
\begin{itemize}
\item For every vector $\mathbf v$, you can visualize $2\mathbf v$ as being twice as
      much length. 
\item For a bivector $B$, $2B$ has twice as much area. 
\item For scalars? Twice as much ``heat''? Or other quantity.
\end{itemize}

\end{frame}

%------------------------------------------------

\begin{frame}{Basic Arithmetic Properties}
\begin{itemize}
\item Scalars are familiar real numbers. 
      \begin{itemize}
            \item The obey the laws of: addition, subtraction, multiplication, etc.
            \item Addition is associative and commutative.
            \item Multiplication of scalars is associative and commutative, and distributes over
                  addition.
      \end{itemize}
\item Vectors
      \begin{itemize}
            \item can be added to each other.
            \item can c be multiplied by scalars in the usual way.
            \item follow associativity and commutativity laws for addition.
            \item follow distributivity laws in multiplication (will be introduced a bit later)
      \end{itemize}
\end{itemize}

\end{frame}
      
%------------------------------------------------



\begin{frame}{Addition of Clifs}
\begin{itemize}
\item Unique property: \textit{any} clif, regardless of grade, can be added together.
      This includes adding scalars to vectors, vectors to bivectors, etc. For instance,
      if $s$ is a scalar, $V$ is a vector and $B$ is a bivector, we can have
      $$C = s + V + B$$
\item Note: addition is allowed (e.g. $s + V$), comparison is not (e.g. $s < V$).
\end{itemize}

\end{frame}
      
%------------------------------------------------



\begin{frame}{Addition Visually}

Bivectors are added edge-to-edge, analogously to vectors' tip-to-tail placement.


\begin{figure}[h]
      \centering
      \begin{subfigure}{0.4\textwidth}
            \centering
            \includegraphics[width=0.5\textwidth]{Figs/add-vectors.png}
            \caption{Addition of vectors}
            \label{fig:first}
      \end{subfigure}
      \begin{subfigure}{0.4\textwidth}
            \centering
            \includegraphics[width=0.5\textwidth]{Figs/add-bivectors.png}
            \caption{Addition of bivectors.}
            \label{fig:first}
      \end{subfigure}
      \caption{Addition of vectors and bivectors.}
      \label{fig:main}
  \end{figure}
\end{frame}
      
%------------------------------------------------

\begin{frame}{Grade Selection}

\begin{itemize}
      \item Since clifs can be additions of objects of many different grades,
            it is nice to have a way to extract only the component you would like.
      \item Thus, we define the following notation: $\langle C \rangle_N$ is the
            grade-N piece of $C$.
      \item For example, $\langle C \rangle_0$ extracts the scalar part of $C$.
      \item Similar to the $\mathcal{R}()$ and $\mathcal{I}()$ operators in complex numbers,
            but the types are unchanged. In complex numbers, the $\mathcal{I}()$ operator
            returns the complex part as a real number. In geometric algebra, the
            grade selection maintains the grade of what's being selected.
\end{itemize}
\end{frame}
      
%------------------------------------------------

\begin{frame}{Multiplication: General Properties}

\begin{itemize}
      \item We write multiplication of two clifs $A$ and $B$ by juxtaposition $AB$.
      \item The geometric product $AB$ follows the laws of:
            \begin{itemize}
                  \item associativity: $(AB)C = A(BC)$
                  \item distributivity: $A(B+C) = AB + AC$
            \end{itemize}
            for all clifs $A$, $B$, and $C$.
      \item Not commutative in general, but the special case of scalar multiplication is.
      \item Multiplying vectors by scalars works as before, and is commutative: $sV = Vs$.
      \item In fact, multiplying any clif by a scalar follows similarly and is also
            commutative. That is, $sC = Cs$ for any scalar $s$ and clif $C$.
\end{itemize}
\end{frame}
      
%------------------------------------------------

\begin{frame}{Multiplication: Vectors by Vectors}

\begin{itemize}
      \item Given two vectors, $P$ and $Q$, we know that the geometric product $PQ$
            exists.
      \item On vectors, we can define two new operations:
            \begin{itemize}
                  \item $P\cdot Q = \frac{PQ + QP}{2}$, where $P$ and $Q$ have grade = 1. \medskip
                  \item $P \land Q = \frac{PQ - QP}{2}$, where $P$ and $Q$ have grade $\leq$ 1.
            \end{itemize}
      \item We call the first one the \textit{dot product}, and the second one the 
            \textit{wedge product}.
      \item Rearranging, we get:
            $$PQ = P\cdot Q + P\land Q$$ where $P$ and $Q$ have grade = 1.
            \begin{itemize}
                  \item Note: It's important to note that this only works for vectors. 
                        It is not the general definition of the geometric product.
                        We will define that later.
            \end{itemize}
\end{itemize}
\end{frame}
      
%------------------------------------------------



\begin{frame}{Properties of Dot Product}

\begin{itemize}
      \item The dot product produces a scalar.
      \item Rotation by 180 degrees does not change the dot product, since 
            $P\cdot P = (-P) \cdot (-P)$. Rotation in perpendicular to $P$
            also does not change the dot product.
      \item Parallel vectors: if $P$ and $Q$ are parallel, symbolized $P \| Q$,
            then $$PQ = QP = P \cdot Q$$.
      \item Perpendicular vectors: If $P\cdot Q = 0$ then the vectors $P$ and $Q$ are
            perpendicular or \textit{orthogonal}, symbolized $P \bot Q$, and 
            $$PQ = -QP = P \land Q$$
\end{itemize}
\end{frame}
      
%------------------------------------------------



\begin{frame}{Properties of Wedge Product}


\begin{itemize}
      \item The wedge product is invariant to rotations of 180 degrees in the PQ
            plane.
      \item If we rotate 180 degrees in a plane perpendicular to the plane, that will
            flip the sign of the wedge product (analogous to the cross product).
      \item The idea of wedge generalized to more than two vectors:
            $$P\land Q\land R = \frac{1}{6}(PQR+QRP+RPQ-RQP-QPR-PRQ)$$
      \item Even more generally,
            $$q_1\land q_2\land q_3\cdots\land q_r = \frac{1}{r}\sum_\pi \text{sign}(\pi)
                  q_{\pi(1)}q_{\pi(2)}q_{\pi(3)}\cdots q_{\pi(r)}$$
            where the sum runs over all permutations $\pi$ and $\text{sign}(\pi)$ is defined as +1 for even permutations and -1 for odd ones.
            \begin{itemize}
                  \item This will be an object of grade $r$ if all vectors are linearly
                        independent, otherwise it will be 0.
            \end{itemize}

\end{itemize}
\end{frame}
      
%------------------------------------------------


\begin{frame}{Some Terminology}


\begin{itemize}
      \item A \textit{blade} is any scalar, vector, or the wedge product of any number of
            vectors.
      \item Any clif that has a definite grade (consists of one single grade)
            is called \textit{homogeneous}. It is either a blade or a sum of blades,
            all of the same grade.
      \item Example: $s + V$ (where $s$ is a scalar and $V$ is a vector) is not homogenous. 
            Does not have a definite grade, and is not a blade.
\end{itemize}
\end{frame}
      
%------------------------------------------------


\begin{frame}{Other Wedge Products}


\begin{itemize}
      \item We can define the wedge product between a blade and another blade. By unpacking
            the wedge product, and getting rid of parentheses, we get:
            $$P \land (Q \land R) = P\land Q\land R$$
            $$(P \land Q) \land R = P\land Q\land R$$
      \item Then, the wedge product contains only the top-grade terms from a geometric
            product:
            If $P = \langle P \rangle_p$ and $Q = \langle Q \rangle_q$, then $P \land Q = \langle PQ \rangle_{p+q}$.
      \item Finally, we need the fact that the wedge product distributes over addition:
            $$V \land (A + B) = V \land A + V \land B$$
      \item Now we have generalized the wedge product to clif wedge clif.

\end{itemize}
\end{frame}
      
%------------------------------------------------


\begin{frame}{Other Dot Products}


\begin{itemize}
      \item Analogously, the dot product of two blades is the lowest-grade part: 
            If $P = \langle P \rangle_p$ and $Q = \langle Q \rangle_q$, then $P \land Q = \langle PQ \rangle_{p-q}$.
      \item Symmetry does not hold in general.
      \item As before, given this definition, we can expand to all clifs knowing that
            $$V \cdot (A+B) = V\cdot A + V \cdot B$$


\end{itemize}
\end{frame}
      
%------------------------------------------------



\begin{frame}{Geometric Product}


\begin{itemize}
      \item<only@1> At last, we can define the geometric product in general!
      \item<only@1> To compute it we must:
            \begin{enumerate}
                  \item Pick a basis
                  \item Project the factors onto that basis
                  \item Multiply everything term-by-term
                  \item Then simplify
            \end{enumerate}
      \item<only@1> Example: $A = \gamma_1 + 2\gamma_1\gamma_2$ and $B = 3\gamma_3 +5\gamma_1\gamma_2\gamma_3$.
            Table~\ref{Tab:Product1} shows the terms.
\only<1>{
\begin{table}
\caption{Term-by-Term Multiplication}\label{Tab:Product1}
\begin{tabularx}{0.6\textwidth}{XXX}
       & $\gamma_1$ & $2\gamma_1\gamma_2$\\
      \midrule
      $3\gamma_3$ & $3\gamma_3\gamma_1$ & $6\gamma_3\gamma_1\gamma_2$ \\
      $5\gamma_1\gamma_2\gamma_3$ & $5\gamma_1\gamma_2\gamma_3\gamma_1$ & $10\gamma_1\gamma_2\gamma_3\gamma_3\gamma_1$ \\
      \bottomrule
\end{tabularx}
\end{table}}
\setcounter{table}{2}
      \item<2> Next, we permute the orthonormal vectors, multiplying by -1 each time we do.
\only<2>{
\begin{table}
\caption{Permuting the Basis Vectors}\label{Tab:Product2}
\begin{tabularx}{0.6\textwidth}{XXX}
       & $\gamma_1$ & $2\gamma_1\gamma_2$\\
      \midrule
      $3\gamma_3$ & $3\gamma_3\gamma_1$ & $6\gamma_1\gamma_2\gamma_3$ \\
      $5\gamma_1\gamma_2\gamma_3$ & $5\gamma_1\gamma_1\gamma_2\gamma_3$ & $-10\gamma_1\gamma_1\gamma_2\gamma_2\gamma_3$ \\
      \bottomrule
\end{tabularx}
\end{table}}



      \item<2> Finally, we simplify, removing any vectors that are the same, since
            a multiplication of orthonormal vectors $\gamma_i\gamma_i = 1$
\only<2>{
\begin{table}
\caption{Simplifying using Orthonormal Vector Properties}\label{Tab:Product3}
\begin{tabularx}{0.6\textwidth}{XXX}
       & $\gamma_1$ & $2\gamma_1\gamma_2$\\
      \midrule
      $3\gamma_3$ & $3\gamma_3\gamma_1$ & $6\gamma_1\gamma_2\gamma_3$ \\
      $5\gamma_1\gamma_2\gamma_3$ & $5\gamma_1\gamma_1\gamma_2\gamma_3$ & $-10\gamma_1\gamma_1\gamma_2\gamma_2\gamma_3$ \\
      \bottomrule
\end{tabularx}
\end{table}}
      \item<2> The final result is $3\gamma_3\gamma_1 + 6\gamma_1\gamma_2\gamma_3 + 5\gamma_1\gamma_1\gamma_2\gamma_3 + -10\gamma_1\gamma_1\gamma_2\gamma_2\gamma_3$.

\end{itemize}
\end{frame}
      
%------------------------------------------------



%------------------------------------------------
\section{Geometric Algebraic Definition of Vectors and Matrices}
%------------------------------------------------

\begin{frame}{Geometric Algebraic Definition of Vectors and Matrices}
    \begin{itemize}
      \item Next, we will define the vector and matrix operations using geometric
            algebra.
      \item The vector operations we defined earlier can be implemented in geometric 
            algebra using grade-1 objects (vectors). The operations follow closely.
      \item Matrix operations depend mostly on the notion of a matrix multiplication.
    \end{itemize}
\end{frame}

%------------------------------------------------

\begin{frame}{Vector Operations in Geometric Algebra}
      \noindent\textbf{Vector Addition}

      Follows from the addition of clif (two vector clifs).
      
      \noindent\textbf{Scalar Multiplication}

      Follows from scalar multiplication of a vector clif.
      
      \noindent\textbf{Vector Subtraction}

      Follows from scalar multiplication and addition of two vector clifs and 
      the scalar -1.
      
      \noindent\textbf{Dot Product}

      Follows from dot product of two vector clifs.

      \noindent \textbf{Cross Product}
      
      The cross product can be defined using the wedge product~\cite{enwiki:1272926547}.
      If we have two vectors (using the standard basis $\textbf{e}$) $\mathbf{u}= u_1\mathbf e_1+u_2\mathbf e_2+u_3\mathbf e_3$ 
      and $\mathbf{v} = v_1\mathbf e_1+v_2\mathbf e_2+v_3\mathbf e_3$,
      then the result is
      $$\mathbf u \land \mathbf v = (u_1v_2 - u_2v_1)(\mathbf e_1 \land \mathbf e_2)+
      (u_3v_1 - u_1v_3)(\mathbf e_3 \land \mathbf e_1)+
      (u_2v_3 - u_3v_2)(\mathbf e_2 \land \mathbf e_3)$$

      Here, the coefficients are the same as before, but the result is a bivector
      instead of a vector.

\end{frame}

%------------------------------------------------

\begin{frame}{Matrix Representation in Geometric Algebra}
      \begin{itemize}
            \item In geometric algebra, a matrix can be represented as a linear transformation on 
            a vector~\cite{Mathoma2019}.
            \item That is, it is a function $\mathbf A : \mathbb{R}^m \to \mathbb{R}^n$, such that $\mathbf A(k\mathbf a + p\mathbf b)
                  = k\mathbf A(\mathbf a) + p\mathbf A(\mathbf b)$.
            \item We'll continue to use the boldface notation to indicate that the
                  function is supposed to be a matrix.
            \item We can use this definition to redefine the idea of a matrix
                  in geometric algebra.
      \end{itemize}
\end{frame}


%------------------------------------------------

\begin{frame}{Matrix Operations in Geometric Algebra}

      \only<1>{\noindent\textbf{Addition}
      
      \noindent The addition of two matrices is calculated as the addition
      of the two functions representing those matrices, since they are linear
      functions. That is,

      $$\mathbf C(\mathbf v) = \mathbf A(\mathbf v) + \mathbf B(\mathbf v)$$
      
      \medskip
      \noindent\textbf{Scalar Multiplication}
      \noindent The scalar multiplication of a real number $s$ and a matrix represented by
      $\mathbf A(v)$
      is denoted as $s\mathbf{A}$ and is computed using the scalar product with the
      resulting vector:

      $$s\mathbf A(v)$$
      
      \medskip
      \noindent\textbf{Subtraction}
      
      \noindent The subtraction of two matrices is denoted as $\mathbf{A} - \mathbf{B}$,
      and is the addition of the scalar multiplication by -1, as before.
      
      $$(\mathbf A - \mathbf B)(v) = \mathbf{A}(v) + (-1)\mathbf{B}(v)$$}
      
      
      \only<2>{\medskip
      \noindent\textbf{Transposition}
      
      \noindent The \textit{transpose} of a matrix $\mathbf A(\mathbf v) : \mathbb{R}^m \to \mathbb{R}^n$ 
      is the matrix $\mathbf{A}^T(\mathbf v) : \mathbb R^n \to \mathbb R^m$ such that the matrices
      behave like the original definition:
      
      $$(\mathbf A^T)_{i,j} = \mathbf A_{j,i}$$
      
      \medskip
      \noindent\textbf{Matrix Multiplication}
      
      \noindent The \textit{matrix multiplication} of matrices $\mathbf A(\mathbf v) : \mathbb R^m \to \mathbb R^p$ 
      and $\mathbf A(\mathbf v) : \mathbb R^p \to \mathbb R^n$, is the
      function composition of the two matrices:
      
      $$(\mathbf A\mathbf B)(\mathbf v) = (\mathbf A \cdot \mathbf B)(\mathbf v)$$}
      
      \medskip
      \noindent\textbf{Other Operations}

      Other described operations, like row operations and submatrix operations,
      depend on matrix multiplication and thus follow from their descriptions earlier 
      in this document. Their sizes will be encoded in the function type.
      
      \end{frame}

%------------------------------------------------
\section{Software Requirements}
%------------------------------------------------

\begin{frame}{Functional Requirements}
      \begin{enumerate}
\item<only@1> The system shall contain an internal representation of geometric algebra.
\item<only@1> The system shall contain a smart constructor for vectors, represented using
      geometric algebra internally.
\item<only@1> The system shall contain a smart constructor for matrices, represented using
      geometric algebra internally.
\item<only@1> The system shall allow the specification of vector, and
      matrix operations with fixed or variable sizes at specification-time.
\item<only@1> The system shall support at least the vector, matrix operations
      defined in this document.
\item<only@1> The system shall allow the generation of documentation for vectors, matrices,
      and generalized clifs.
\item<only@1> The system shall allow the generation of code to perform the operations
      for vectors, matrices, and generalized clifs.
\item<only@1> The system shall ensure or check the validity of operations at specification 
      time, generation-time, or runtime, as appropriate.
\item<only@2> For a given problem specified using this extension, generated documents shall 
      logically match the generated code for performing the given operations.
\item<only@2> The addition of these new features to Drasil shall not break any existing
      examples.
      \end{enumerate}
\end{frame}

%------------------------------------------------

\begin{frame}{References}
    \tiny
    \bibliography {../../refs/References}
    \bibliographystyle{plainnat}
\end{frame}

%------------------------------------------------

\begin{frame}
    \Huge{\centerline{\textbf{The End}}}
\end{frame}

%----------------------------------------------------------------------------------------

\end{document}