% THIS DOCUMENT IS TAILORED TO REQUIREMENTS FOR SCIENTIFIC COMPUTING.  IT SHOULDN'T
% BE USED FOR NON-SCIENTIFIC COMPUTING PROJECTS
\documentclass[12pt]{article}

\usepackage{amsmath, mathtools}
\usepackage{amsfonts}
\usepackage{amssymb}
\usepackage{graphicx}
\usepackage{colortbl}
\usepackage{xr}
\usepackage{hyperref}
\usepackage{longtable}
\usepackage{xfrac}
\usepackage{tabularx}
\usepackage{float}
\usepackage{siunitx}
\usepackage{booktabs}
\usepackage{caption}
\usepackage{pdflscape}
\usepackage{afterpage}

\usepackage[round]{natbib}

%\usepackage{refcheck}

\hypersetup{
    bookmarks=true,         % show bookmarks bar?
      colorlinks=true,       % false: boxed links; true: colored links
    linkcolor=red,          % color of internal links (change box color with linkbordercolor)
    citecolor=green,        % color of links to bibliography
    filecolor=magenta,      % color of file links
    urlcolor=cyan           % color of external links
}

\input{../Comments}
%% Common Parts

\newcommand{\progname}{Drasil Matrix, Vector and Tensor Extension} % PUT YOUR PROGRAM NAME HERE
\newcommand{\authname}{Christopher W. Schankula} % AUTHOR NAMES                  

\usepackage{hyperref}
    \hypersetup{colorlinks=true, linkcolor=blue, citecolor=blue, filecolor=blue,
                urlcolor=blue, unicode=false}
    \urlstyle{same}
                                


% For easy change of table widths
\newcommand{\colZwidth}{1.0\textwidth}
\newcommand{\colAwidth}{0.13\textwidth}
\newcommand{\colBwidth}{0.82\textwidth}
\newcommand{\colCwidth}{0.1\textwidth}
\newcommand{\colDwidth}{0.05\textwidth}
\newcommand{\colEwidth}{0.8\textwidth}
\newcommand{\colFwidth}{0.17\textwidth}
\newcommand{\colGwidth}{0.5\textwidth}
\newcommand{\colHwidth}{0.28\textwidth}

% Used so that cross-references have a meaningful prefix
\newcounter{defnum} %Definition Number
\newcommand{\dthedefnum}{GD\thedefnum}
\newcommand{\dref}[1]{GD\ref{#1}}
\newcounter{datadefnum} %Datadefinition Number
\newcommand{\ddthedatadefnum}{DD\thedatadefnum}
\newcommand{\ddref}[1]{DD\ref{#1}}
\newcounter{theorynum} %Theory Number
\newcommand{\tthetheorynum}{TM\thetheorynum}
\newcommand{\tref}[1]{TM\ref{#1}}
\newcounter{tablenum} %Table Number
\newcommand{\tbthetablenum}{TB\thetablenum}
\newcommand{\tbref}[1]{TB\ref{#1}}
\newcounter{assumpnum} %Assumption Number
\newcommand{\atheassumpnum}{A\theassumpnum}
\newcommand{\aref}[1]{A\ref{#1}}
\newcounter{goalnum} %Goal Number
\newcommand{\gthegoalnum}{GS\thegoalnum}
\newcommand{\gsref}[1]{GS\ref{#1}}
\newcounter{instnum} %Instance Number
\newcommand{\itheinstnum}{IM\theinstnum}
\newcommand{\iref}[1]{IM\ref{#1}}
\newcounter{reqnum} %Requirement Number
\newcommand{\rthereqnum}{R\thereqnum}
\newcommand{\rref}[1]{R\ref{#1}}
\newcounter{nfrnum} %NFR Number
\newcommand{\rthenfrnum}{NFR\thenfrnum}
\newcommand{\nfrref}[1]{NFR\ref{#1}}
\newcounter{lcnum} %Likely change number
\newcommand{\lthelcnum}{LC\thelcnum}
\newcommand{\lcref}[1]{LC\ref{#1}}

\usepackage{fullpage}

\newcommand{\deftheory}[9][Not Applicable]
{
\newpage
\noindent \rule{\textwidth}{0.5mm}

\paragraph{RefName: } \textbf{#2} \phantomsection 
\label{#2}

\paragraph{Label:} #3

\noindent \rule{\textwidth}{0.5mm}

\paragraph{Equation:}

#4

\paragraph{Description:}

#5

\paragraph{Notes:}

#6

\paragraph{Source:}

#7

\paragraph{Ref.\ By:}

#8

\paragraph{Preconditions for \hyperref[#2]{#2}:}
\label{#2_precond}

#9

\paragraph{Derivation for \hyperref[#2]{#2}:}
\label{#2_deriv}

#1

\noindent \rule{\textwidth}{0.5mm}

}

\begin{document}

\title{Software Requirements Specification for \progname} 
\author{\authname}
\date{\today}
	
\maketitle

~\newpage

\pagenumbering{roman}

\tableofcontents

~\newpage

\section*{Revision History}

\begin{tabularx}{\textwidth}{p{3cm}p{2cm}X}
\toprule {\bf Date} & {\bf Version} & {\bf Notes}\\
\midrule
January 23rd & 1.0 & Initial Work on Document for Presentation\\
Date 2 & 1.1 & Notes\\
\bottomrule
\end{tabularx}

~\newpage

\section{Reference Material}

This section records information for easy reference.

\subsection{Table of Symbols}

The table that follows summarizes the symbols used in this document along with
their units.  The choice of symbols was made to be consistent with the heat
transfer literature and with existing documentation for solar water heating
systems.  The symbols are listed in alphabetical order.

\renewcommand{\arraystretch}{1.2}
%\noindent \begin{tabularx}{1.0\textwidth}{l l X}
\noindent \begin{longtable*}{l l p{12cm}} \toprule
\textbf{symbol} & \textbf{unit} & \textbf{description}\\
\midrule 
$A_C$ & \si[per-mode=symbol] {\square\metre} & coil surface area
\\
$A_\text{in}$ & \si[per-mode=symbol] {\square\metre} & surface area over 
which heat is transferred in
\\ 
\bottomrule
\end{longtable*}
\plt{Use your problems actual symbols.  The si package is a good idea to use for
  units.}

\subsection{Abbreviations and Acronyms}

\renewcommand{\arraystretch}{1.2}
\begin{tabular}{l l} 
  \toprule		
  \textbf{symbol} & \textbf{description}\\
  \midrule 
  A & Assumption\\
  DD & Data Definition\\
  GD & General Definition\\
  GS & Goal Statement\\
  IM & Instance Model\\
  LC & Likely Change\\
  PS & Physical System Description\\
  R & Requirement\\
  SRS & Software Requirements Specification\\
  \progname{} & \plt{put an expanded version of your program name here (as appropriate)}\\
  TM & Theoretical Model\\
  \bottomrule
\end{tabular}\\

\plt{Add any other abbreviations or acronyms that you add}

\subsection{Mathematical Notation}


\newpage

\pagenumbering{arabic}

\section{Introduction}
This introduction section states the purpose of this document, the scope of 
the requirements, the characteristics of the intended reader, an overview of the
Drasil project, and describes the organization of the rest of the document.

\subsection{Purpose of Document}
The purpose of this document is to document the necessary mathematical background for,
and the software requirements of, an extension of the Drasil project encoding tensor,
vector, and matrix operations. It is intended to allow the different stakeholders to
communicate about and iterate on the project in a formal way. The document will likely
be updated throughout the project, and the changes will be recorded in the Revision
History table.

\subsection{Scope of Requirements} 
The scope of the requirements will be related to the addition of tensors, vectors,
and matrices to the Drasil project.

\subsection{Characteristics of Intended Reader}\label{sec_IntendedReader}

\subsection{Overview of Drasil Project}
Drasil is ``a framework a framework for generating all of the software artifacts from 
a stable knowledge base, focusing currently on scientific software''. The framework
is written in Haskell and allows generation of Software Requirements Specifications,
Python, Java, C-Sharp, and C++ code, README files, and Makefiles.

\subsection{Organization of Document}
The rest of the document is organized as follows: Section~\ref{Sec:RelatedWork} 
presents related work in the area of tensors, Section~\ref{Sec:MathematicalDefinitions}
presents the mathematical definitions, transformation rules and allowed operations
on tensors, vectors, and matrices. Then, Section~\ref{Sec:VecMatAsTensors} describes
how vectors and matrices can be defined as special cases of tensors. Finally, 
Section~\ref{Sec:Req} presents general requirements for the system extension as well as
a set of ``test-driven'' requirements, denoting some example scientific problems
to be encoded and solved using this new system, which will provide oracles with which
to test its correctness.


\section{Related Work}\label{Sec:RelatedWork}

\subsection{Tensors in Haskell}

\subsection{Tensors in Other Languages}

\section{Mathematical Definitions}\label{Sec:MathematicalDefinitions}
This section provides mathematical definitions of tensors, vectors, and matrices.
This includes terminology needed for each, the notation used to describe each one,
the transformation rules governing each one, and the allowed operations we are 
targeting for the software. Note that vectors and matrices will be defined here without
relying on their definition as a tensor; Section~\ref{Sec:VecMatAsTensors} will redefine
them using tensors.

\subsection{Tensors}
A \textit{tensor} is a mathematical 

\subsubsection{Terms}

\subsubsection{Data Storage}

\subsubsection{Einstein Summation Notation}

\subsubsection{Transformation Rules}

\subsubsection{Allowed Operations}


\subsection{Vectors}

\subsubsection{Terms}

\subsubsection{Notation}

\subsubsection{Allowed Operations}


\subsection{Matrices}

\subsubsection{Terms}

\subsubsection{Notation}

\subsubsection{Allowed Operations}


\section{Vectors and Matrices Defined as Tensors}\label{Sec:VecMatAsTensors}

\subsection{Vectors}

\subsubsection{Tensor Definition of Vectors}

\subsubsection{Allowed Operations as Tensors}

\subsection{Matrices}

\subsubsection{Tensor Definition of Vectors}

\subsubsection{Allowed Operations as Tensors}


\section{Requirements}\label{Sec:Req}

\subsection{General Requirements}

\subsection{Test-Driven Requirements}










\newpage

\bibliographystyle {plainnat}
\bibliography {../../refs/References}

\end{document}