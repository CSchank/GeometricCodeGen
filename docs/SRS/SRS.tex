%----------------------------------------------------------------------------------------
%    PACKAGES AND THEMES
%----------------------------------------------------------------------------------------

\documentclass[aspectratio=169,xcolor=dvipsnames]{beamer}
\usetheme{SimpleDarkBlue}

\usepackage{hyperref}
\usepackage{graphicx} % Allows including images
\usepackage{booktabs} % Allows the use of \toprule, \midrule and \bottomrule in tables

%----------------------------------------------------------------------------------------
%    TITLE PAGE
%----------------------------------------------------------------------------------------

\title{Drasil Geometric Algebra Extension}
\subtitle{Overview of Requirements}

\author{Christopher Schankula}

\institute
{
    Department of Computing \& Software \\
    McMaster University % Your institution for the title page
}
\date{\today} % Date, can be changed to a custom date

%----------------------------------------------------------------------------------------
%    PRESENTATION SLIDES
%----------------------------------------------------------------------------------------

\begin{document}

\begin{frame}
    % Print the title page as the first slide
    \titlepage
\end{frame}

\begin{frame}{Overview}
    % Throughout your presentation, if you choose to use \section{} and \subsection{} commands, these will automatically be printed on this slide as an overview of your presentation
    \tableofcontents

    The purpose of this slide deck is to define the objects and operations that will
    be implemented in the project. We will begin with a discussion about vectors and
    matrices, then generalize with geometric algebra. Finally, we will redefine the
    operations using geometric algebra.
\end{frame}

%------------------------------------------------
\section{Einstein Summation Notation}
%------------------------------------------------

\begin{frame}{Einstein Summation Notation}
    \begin{itemize}
      \item Used to describe repeated summations and multiplications in a compact notation.
      \item They behave according to \textbf{four rules}~\cite{Khan2023}.
            \begin{enumerate}
            \item Any twice-repeated index in a single term is summed over.
            \item A twice-repeated index is called a dummy index; a once-repeated 
                  index is called a free index.
            \item No index may occur 3 or more times in a given term.
            \item In an equation with Einstein notation, the free indices on both sides must
                  match.
            \end{enumerate}
    \end{itemize}
\end{frame}

\begin{frame}{Einstein Summation Notation Rules}
      Here are the rules with some more explanations~\cite{Khan2023}:
      \begin{enumerate}
      \item<only@1> \textit{Any twice-repeated index in a single term is summed over.}\\
            For example, $a_{ij}b_i$ represents the term $\sum_{i=1}^j a_{ij}b_i$.
      \item<only@1> \textit{A twice-repeated index is called a dummy index; a once-repeated 
            index is called a free index.}\\
            In the example above, $i$ is a dummy index \textemdash~ it can be renamed
            however you would like. However, $j$ is a free index and has restrictions on naming.
      \item<only@1> \textit{No index may occur 3 or more times in a given term.}\\
            For example, $a_{ii}b_i$ is not legal.
      \item<only@2>{\textit{In an equation with Einstein notation, the free indices on both sides must
            match.}\\

            \noindent Some examples of correctly-formed equations:

            \begin{itemize}
            \item $x_i = a_{i,j}b_i$ is valid because $i$ is free on both the LHS and RHS
            \item $a_i = A_{k,i}B_{k,j}x_j + C_{i,k}u_k$ is valid because $i$ is a free variable on
                  the LHS, and in every term it is the free variable on the RHS.
            \end{itemize}

            \noindent Some examples of incorrectly-formed equations:

            \begin{itemize}
            \item $x_i = A_{j,i}$ is invalid because $i$ is the only free variable on the LHS, but
                  $i$ and $j$ are both free on the RHS.
            \item $x_j = A{i,k}u_k$ is invalid because $j$ is free on the LHS, but $i$ is free on 
                  the RHS.
            \item $x_i = A_{i,k}u_k + c_j$ is invalid because $i$ is free on the LHS, but on the 
                  RHS, one term has $i$ free while the other term has $j$ free.
            \end{itemize}}

      \end{enumerate}
      We will use this convention from now on.
  \end{frame}

%------------------------------------------------



%------------------------------------------------
\section{Vectors}
%------------------------------------------------

\begin{frame}{Vectors}
    \begin{itemize}
      \item 
    \end{itemize}
\end{frame}

%------------------------------------------------

\begin{frame}{Blocks of Highlighted Text}
    In this slide, some important text will be \alert{highlighted} because it's important. Please, don't abuse it.

    \begin{block}{Block}
        Sample text
    \end{block}

    \begin{alertblock}{Alertblock}
        Sample text in red box
    \end{alertblock}

    \begin{examples}
        Sample text in green box. The title of the block is ``Examples".
    \end{examples}
\end{frame}

%------------------------------------------------

\begin{frame}{Multiple Columns}
    \begin{columns}[c] % The "c" option specifies centered vertical alignment while the "t" option is used for top vertical alignment

        \column{.45\textwidth} % Left column and width
        \textbf{Heading}
        \begin{enumerate}
            \item Statement
            \item Explanation
            \item Example
        \end{enumerate}

        \column{.45\textwidth} % Right column and width
        Lorem ipsum dolor sit amet, consectetur adipiscing elit. Integer lectus nisl, ultricies in feugiat rutrum, porttitor sit amet augue. Aliquam ut tortor mauris. Sed volutpat ante purus, quis accumsan dolor.

    \end{columns}
\end{frame}

%------------------------------------------------
\section{Matrices}
%------------------------------------------------

\begin{frame}{Table}
    \begin{table}
        \begin{tabular}{l l l}
            \toprule
            \textbf{Treatments} & \textbf{Response 1} & \textbf{Response 2} \\
            \midrule
            Treatment 1         & 0.0003262           & 0.562               \\
            Treatment 2         & 0.0015681           & 0.910               \\
            Treatment 3         & 0.0009271           & 0.296               \\
            \bottomrule
        \end{tabular}
        \caption{Table caption}
    \end{table}
\end{frame}

%------------------------------------------------

\begin{frame}{Theorem}
    \begin{theorem}[Mass--energy equivalence]
        $E = mc^2$
    \end{theorem}
\end{frame}

%------------------------------------------------

\begin{frame}{Figure}
    Uncomment the code on this slide to include your own image from the same directory as the template .TeX file.
    %\begin{figure}
    %\includegraphics[width=0.8\linewidth]{test}
    %\end{figure}
\end{frame}

%------------------------------------------------
\section{Geometric Algebra}
%------------------------------------------------


%------------------------------------------------

\begin{frame}[fragile] % Need to use the fragile option when verbatim is used in the slide
    \frametitle{Citation}
    An example of the \verb|\cite| command to cite within the presentation:\\~

    This statement requires citation \cite{p1}.
\end{frame}

%------------------------------------------------

%------------------------------------------------
\section{Geometric Algebraic Definition of Vectors and Matrices}
%------------------------------------------------

\begin{frame}{References}
    \footnotesize
    \bibliography {../../refs/References}
    \bibliographystyle{apalike}
\end{frame}

%------------------------------------------------

\begin{frame}
    \Huge{\centerline{\textbf{The End}}}
\end{frame}

%----------------------------------------------------------------------------------------

\end{document}