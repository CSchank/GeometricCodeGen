% THIS DOCUMENT IS TAILORED TO REQUIREMENTS FOR SCIENTIFIC COMPUTING.  IT SHOULDN'T
% BE USED FOR NON-SCIENTIFIC COMPUTING PROJECTS
\documentclass[12pt]{article}

\usepackage{amsmath, mathtools}
\usepackage{amsfonts}
\usepackage{amssymb}
\usepackage{graphicx}
\usepackage{colortbl}
\usepackage{xr}
\usepackage{hyperref}
\usepackage{longtable}
\usepackage{xfrac}
\usepackage{tabularx}
\usepackage{float}
\usepackage{siunitx}
\usepackage{booktabs}
\usepackage{caption}
\usepackage{pdflscape}
\usepackage{afterpage}
\usepackage{enumitem}

\usepackage[round]{natbib}

%\usepackage{refcheck}

\hypersetup{
    bookmarks=true,         % show bookmarks bar?
      colorlinks=true,       % false: boxed links; true: colored links
    linkcolor=red,          % color of internal links (change box color with linkbordercolor)
    citecolor=green,        % color of links to bibliography
    filecolor=magenta,      % color of file links
    urlcolor=cyan           % color of external links
}

\input{../Comments}
%% Common Parts

\newcommand{\progname}{Drasil Matrix, Vector and Tensor Extension} % PUT YOUR PROGRAM NAME HERE
\newcommand{\authname}{Christopher W. Schankula} % AUTHOR NAMES                  

\usepackage{hyperref}
    \hypersetup{colorlinks=true, linkcolor=blue, citecolor=blue, filecolor=blue,
                urlcolor=blue, unicode=false}
    \urlstyle{same}
                                


% For easy change of table widths
\newcommand{\colZwidth}{1.0\textwidth}
\newcommand{\colAwidth}{0.13\textwidth}
\newcommand{\colBwidth}{0.82\textwidth}
\newcommand{\colCwidth}{0.1\textwidth}
\newcommand{\colDwidth}{0.05\textwidth}
\newcommand{\colEwidth}{0.8\textwidth}
\newcommand{\colFwidth}{0.17\textwidth}
\newcommand{\colGwidth}{0.5\textwidth}
\newcommand{\colHwidth}{0.28\textwidth}

% Used so that cross-references have a meaningful prefix
\newcounter{defnum} %Definition Number
\newcommand{\dthedefnum}{GD\thedefnum}
\newcommand{\dref}[1]{GD\ref{#1}}
\newcounter{datadefnum} %Datadefinition Number
\newcommand{\ddthedatadefnum}{DD\thedatadefnum}
\newcommand{\ddref}[1]{DD\ref{#1}}
\newcounter{theorynum} %Theory Number
\newcommand{\tthetheorynum}{TM\thetheorynum}
\newcommand{\tref}[1]{TM\ref{#1}}
\newcounter{tablenum} %Table Number
\newcommand{\tbthetablenum}{TB\thetablenum}
\newcommand{\tbref}[1]{TB\ref{#1}}
\newcounter{assumpnum} %Assumption Number
\newcommand{\atheassumpnum}{A\theassumpnum}
\newcommand{\aref}[1]{A\ref{#1}}
\newcounter{goalnum} %Goal Number
\newcommand{\gthegoalnum}{GS\thegoalnum}
\newcommand{\gsref}[1]{GS\ref{#1}}
\newcounter{instnum} %Instance Number
\newcommand{\itheinstnum}{IM\theinstnum}
\newcommand{\iref}[1]{IM\ref{#1}}
\newcounter{reqnum} %Requirement Number
\newcommand{\rthereqnum}{R\thereqnum}
\newcommand{\rref}[1]{R\ref{#1}}
\newcounter{nfrnum} %NFR Number
\newcommand{\rthenfrnum}{NFR\thenfrnum}
\newcommand{\nfrref}[1]{NFR\ref{#1}}
\newcounter{lcnum} %Likely change number
\newcommand{\lthelcnum}{LC\thelcnum}
\newcommand{\lcref}[1]{LC\ref{#1}}

\usepackage{fullpage}

\newcommand{\deftheory}[9][Not Applicable]
{
\newpage
\noindent \rule{\textwidth}{0.5mm}

\paragraph{RefName: } \textbf{#2} \phantomsection 
\label{#2}

\paragraph{Label:} #3

\noindent \rule{\textwidth}{0.5mm}

\paragraph{Equation:}

#4

\paragraph{Description:}

#5

\paragraph{Notes:}

#6

\paragraph{Source:}

#7

\paragraph{Ref.\ By:}

#8

\paragraph{Preconditions for \hyperref[#2]{#2}:}
\label{#2_precond}

#9

\paragraph{Derivation for \hyperref[#2]{#2}:}
\label{#2_deriv}

#1

\noindent \rule{\textwidth}{0.5mm}

}

\begin{document}

\title{Software Requirements Specification for \progname} 
\author{\authname}
\date{\today}
	
\maketitle

~\newpage

\pagenumbering{roman}

\tableofcontents

~\newpage

\section*{Revision History}

\begin{tabularx}{\textwidth}{p{3cm}p{2cm}X}
\toprule {\bf Date} & {\bf Version} & {\bf Notes}\\
\midrule
January 23rd & 1.0 & Initial Work on Document for Presentation\\
Date 2 & 1.1 & Notes\\
\bottomrule
\end{tabularx}

~\newpage

\section{Reference Material}

This section records information for easy reference.

\subsection{Table of Symbols}

The table that follows summarizes the symbols used in this document along with
their units.  The choice of symbols was made to be consistent with the heat
transfer literature and with existing documentation for solar water heating
systems.  The symbols are listed in alphabetical order.

\renewcommand{\arraystretch}{1.2}
%\noindent \begin{tabularx}{1.0\textwidth}{l l X}
\noindent \begin{longtable*}{l l p{12cm}} \toprule
\textbf{symbol} & \textbf{unit} & \textbf{description}\\
\midrule 
$A_C$ & \si[per-mode=symbol] {\square\metre} & coil surface area
\\
$A_\text{in}$ & \si[per-mode=symbol] {\square\metre} & surface area over 
which heat is transferred in
\\ 
\bottomrule
\end{longtable*}
\plt{Use your problems actual symbols.  The si package is a good idea to use for
  units.}

\subsection{Terminology}
\begin{itemize}
\item \textbf{Specification-time}: A property that holds true when a specification is
     created in the Drasil framework.
\item \textbf{Generation-time}: A property that holds true or is checked when the 
     code/documentation is generated by the Drasil system. 
\item \textbf{Runtime}: A property that holds true or is checked when the generated 
     code is run.
\end{itemize}

\subsection{Abbreviations and Acronyms}

\renewcommand{\arraystretch}{1.2}
\begin{tabular}{l l} 
  \toprule		
  \textbf{symbol} & \textbf{description}\\
  \midrule 
  A & Assumption\\
  DD & Data Definition\\
  GD & General Definition\\
  GS & Goal Statement\\
  IM & Instance Model\\
  LC & Likely Change\\
  PS & Physical System Description\\
  R & Requirement\\
  SRS & Software Requirements Specification\\
  \progname{} & \plt{put an expanded version of your program name here (as appropriate)}\\
  TM & Theoretical Model\\
  \bottomrule
\end{tabular}\\

\plt{Add any other abbreviations or acronyms that you add}

\subsection{Mathematical Notation}


\newpage

\pagenumbering{arabic}

\section{Introduction}
This introduction section states the purpose of this document, the scope of 
the requirements, the characteristics of the intended reader, an overview of the
Drasil project, and describes the organization of the rest of the document.

\subsection{Purpose of Document}
The purpose of this document is to document the necessary mathematical background for,
and the software requirements of, an extension of the Drasil project encoding tensor,
vector, and matrix operations. It is intended to allow the different stakeholders to
communicate about and iterate on the project in a formal way. The document will likely
be updated throughout the project, and the changes will be recorded in the Revision
History table.

\subsection{Scope of Requirements} 
The scope of the requirements will be related to the addition of tensors, vectors,
and matrices to the Drasil project.

\subsection{Characteristics of Intended Reader}\label{sec_IntendedReader}

\subsection{Overview of Drasil Project}
Drasil is ``a framework a framework for generating all of the software artifacts from 
a stable knowledge base, focusing currently on scientific software''. The framework
is written in Haskell and allows generation of Software Requirements Specifications,
Python, Java, C-Sharp, and C++ code, README files, and Makefiles.

\subsection{Organization of Document}
The rest of the document is organized as follows: Section~\ref{Sec:RelatedWork} 
presents related work in the area of tensors, Section~\ref{Sec:MathematicalDefinitions}
presents the mathematical definitions, transformation rules and allowed operations
on tensors, vectors, and matrices. Then, Section~\ref{Sec:VecMatAsTensors} describes
how vectors and matrices can be defined as special cases of tensors. Finally, 
Section~\ref{Sec:Req} presents general requirements for the system extension as well as
a set of ``test-driven'' requirements, denoting some example scientific problems
to be encoded and solved using this new system, which will provide oracles with which
to test its correctness.


\section{Related Work}\label{Sec:RelatedWork}

\subsection{Tensors in Haskell}

\subsection{Tensors in Other Languages}

\section{Mathematical Definitions}\label{Sec:MathematicalDefinitions}
This section provides mathematical definitions of tensors, vectors, and matrices.
This includes terminology needed for each, the notation used to describe each one,
the transformation rules governing each one, and the allowed operations we are 
targeting for the software. Note that vectors and matrices will be defined here without
relying on their definition as a tensor; Section~\ref{Sec:VecMatAsTensors} will redefine
them using tensors.

\subsection{Einstein Summation Notation}

Einstein summation notation is used to describe repeated summations and multiplications
in a compact notation. They behave according to the following four rules~\citep{Khan2023}:

\medskip
\noindent\textit{Rule 1: Any twice-repeated index in a single term is summed over.}

\noindent For example, writing the term $a_{i,j}b_i$ represents the sum 
$\sum_{i=1}^n a_{i,j}b_i$. With Einstein notation, we can omit the sum notation.

\medskip
\noindent\textit{Rule 2: A twice-repeated index is called a dummy index; a 
once-repeated index is called a free index.}

\noindent In the example above, $i$ is a dummy index \textemdash~it can be renamed
however you would like and it still means $1\ldots n$. However, $j$ is a free index
and has restrictions on naming.

\medskip
\noindent\textit{Rule 3: No index may occur 3 or more times in a given term.}

\noindent For example, the term $a_{i,i}b_i$ is not legal.

\medskip
\noindent\textit{Rule 4: In an equation involving Einstein notation, the free indices
on both sides must match.}

\noindent Some examples of correctly-formed equations:

\begin{itemize}
\item $x_i = a_{i,j}b_i$ is valid because $i$ is free on both the LHS and RHS
\item $a_i = A_{k,i}B_{k,j}x_j + C_{i,k}u_k$ is valid because $i$ is a free variable on
      the LHS, and in every term it is the free variable on the RHS.
\end{itemize}

\noindent Some examples of incorrectly-formed equations:

\begin{itemize}
  \item $x_i = A_{j,i}$ is invalid because $i$ is the only free variable on the LHS, but
      $i$ and $j$ are both free on the RHS.
  \item $x_j = A{i,k}u_k$ is invalid because $j$ is free on the LHS, but $i$ is free on 
      the RHS.
  \item $x_i = A_{i,k}u_k + c_j$ is invalid because $i$ is free on the LHS, but on the 
      RHS, one term has $i$ free while the other term has $j$ free.
  \end{itemize}

The remainder of this document will use Einstein notation to enhance brevity and clarity.

\subsection{Vectors}
A vector is a quantity having a magnitude and a direction~\cite{Wikipedia_Vectors_2024}.
They are used in mathematics and physics problems, especially those involving 

\subsubsection{Terms}
\begin{itemize}
\item \textbf{Component}: An individual element of a vector.
\item \textbf{Magnitude}: The size or ``length'' of a vector.
\item \textbf{Direction}: The way a vector points in space.
\item \textbf{Basis Vector}: 
\end{itemize}

\subsubsection{Notation}
We will write vectors using the boldface, such as $\mathbf{V}$. An example of a vector with 
three elements is $\mathbf{a} = \begin{bmatrix}a_1 & a_2 & a_3 \end{bmatrix}$. This is
called a \textit{row vector}. An example of a column vector is $\mathbf{b} = 
\begin{bmatrix}b_1 \\ b_2 \\ b_3 \end{bmatrix}$.

\subsubsection{Allowed Operations} 

This subsection defines the allowed operations on vectors.

\noindent \textbf{Equality}

Two vectors are said to be equal if they have the same magnitude and 
direction~\citep{Wikipedia_EuclidVectors_2025}. This can also be defined by
equating their components, i.e. 

$\mathbf{a}_i = \mathbf{b}_i\ \forall~i = 1\ldots n$

\noindent\textbf{Vector Addition}

\noindent Two vectors with the same size can be added together. Adding them adds their
components together. That is,

$\mathbf{a} + \mathbf{b} = \mathbf{a}_i + \mathbf{b}_i$

\noindent\textbf{Scalar Multiplication}
The scalar multiplication of a vector is equivalent to multiplying the magnitude
by that number. Equivalently, it is the same as multiplying each component by the
number. That is, a vector $\mathbf{a}$ scaled by the real number $r$ is denoted
$r\mathbf{a}$ and is defined as:

$(r\mathbf{a})_i = r\mathbf{a}_i, 1 \leq i \leq n$

\noindent \textbf{Dot Product}

The \textit{dot product} of two vectors each of size $n$ is represented by

$\mathbf{a} \cdot \mathbf{b} = \mathbf{a}_i\mathbf{b}_i$.

\noindent \textbf{Cross Product}

The \textit{cross product} $\mathbf{c}$ of 3-dimensional vectors $\mathbf{a}$ and
$\mathbf{b}$ is denoted as $\mathbf{c} = \mathbf{a} \times 
\mathbf{b}$, and is defined by Einstein notation as~\citep{ContinuumMechanics2012}:

$c_i = \epsilon_{i,j,k}\mathbf{a}_j\mathbf{b}_k$, where $\epsilon_{1,2,3}=
\epsilon_{2,3,1}=\epsilon_{3,1,2} = 1$ and $\epsilon_{3,2,1} = \epsilon_{2,1,3}
= \epsilon_{1,3,2} = -1$.


\subsection{Matrices}
A \textit{matrix} is a rectangular table of numbers, symbols, or expressions
with entries arranged in rows and columns~\citep{Wikipedia_Matrix_2025}. For
example, the following is a 2-by-3 matrix: 

$$\mathbf{A} = 
\begin{bmatrix}
  a_{1,1} & a_{1,2} & a_{1,3} \\
  a_{2,1} & a_{2,2} & a_{2,3}
 \end{bmatrix}$$

\subsubsection{Terms}

\begin{itemize}
\item \textbf{Size}: The number of rows and columns in a matrix. A matrix with
      $m$ rows and $n$ columns is called an $m \times n$ matrix, or an $m$-by-$n$ matrix.
\item \textbf{Dimensions}: In the above example, $m$ and $n$ are called the matrix's
      \textit{dimensions}.
\item \textbf{Entries}: An individual element in the matrix.
\end{itemize}

\subsubsection{Notation}

By convention, we refer to matrices using an uppercase boldface name and
their components using the lowercase equivalent without using boldface. The
symbol $\mathbf{a}_{i,j}$ represents the \textit{entry} at the $i$th row
and $j$th column of the matrix.

\subsubsection{Allowed Operations}
The following is the list of allowed operations on matrices:

\medskip
\noindent\textbf{Addition}

\noindent The addition of two $m \times n$ matrices is calculated as:

$(\mathbf A + \mathbf B)_{i,j} = \mathbf{A}_{i,j} + \mathbf{B}_{i,j}, 1 \leq i \leq m, 1 \leq j \leq n$

\medskip
\noindent\textbf{Scalar Multiplication}
\noindent The scalar multiplication of a real number $c$ and a matrix $\mathbf{A}$ 
is denoted as $c\mathbf{A}$ and is computed as:

$(c\mathbf A)_{i,j} = c\mathbf{A}_{i,j}, 1 \leq i \leq m, 1 \leq j \leq n$

\medskip
\noindent\textbf{Subtraction}

\noindent The subtraction of two $m \times n$ matrices is denoted as $\mathbf{A} - \mathbf{B}$,
and is the addition of the scalar multiplication by -1:

$(\mathbf A - \mathbf B)_{i,j} = \mathbf{A}_{i,j} + (-1)\mathbf{B}_{i,j}, 1 \leq i \leq m, 1 \leq j \leq n$


\medskip
\noindent\textbf{Transposition}

\noindent The \textit{transpose} of an $m\times n$ matrix is the $n\times m$ matrix $\mathbf{A}^T$
formed by turning rows into columns and vice versa:

$(\mathbf A^T)_{i,j} = \mathbf A_{j,i}$.

\medskip
\noindent\textbf{Matrix Multiplication}

\noindent The \textit{matrix multiplication} of matrices $\mathbf A$ and $\mathbf B$, is 
defined when matrix $\mathbf A$ is of size $m \times p$ and matrix $\mathbf B$ is of 
size $p \times n$. Then, the resulting matrix $\mathbf A\mathbf B$ is a matrix of size 
$m \times n$ such that

$(\mathbf A\mathbf B)_{i,j} = a_{i,r}b_{r,j}$


\medskip
\noindent\textbf{Row Operations}
TODO

\begin{enumerate}
\item Row addition
\item Row multiplication
\item Row switching
\end{enumerate}


\medskip
\noindent\textbf{Submatrix}
TODO 

\noindent A \textit{submatrix} is obtained by deleting any collection or rows and/or columns.


\subsection{Tensors}
A \textit{tensor} is an ``algebraic object that describes a multilinear
relationship between sets of algebraic objects such as vectors, scalars
and even other tensors''~\citep{Wikipedia_Tensors_2025}. While tensors
are considered to be independent of any basis, they are often referred
to by their components in a particular basis; in this case they are
stored as a multidimensional array. We will take advantage of this 
definition since it is a straightforward way to define and store tensors.

\subsubsection{Terms}
This subsection denotes common terminology used to describe tensors:

\begin{itemize}
\item \textbf{Index}: 
\item \textbf{Components}: The numbers in the multidimensional array representing a
      tensor (see below).
\item \textbf{Order}: The total number of indices needed to uniquely identify each of the 
      components of a tensor.
\item \textbf{Basis vector}: A set $B$ of vectors in a vector space $V$ such that every 
      vector in $V$ can be written as a unique linear combination of the vectors 
      in $B$~\citep{Wikipedia_Basis_2025}
\item \textbf{Coordinate Transformation}: 
\end{itemize}

\subsubsection{Multidimensional Array Representation}
A common and convenient representation of a tensor for a given basis is using a 
multidimensional array to store the tensor. These can be thought of as higher-dimensional
matrices. For instance, the tensor $T_{i,j}$ is
the tensor of order two; it has two indices $i$ and $j$. This is also called a
second-order tensor. If the indices $i$ and $j$ run from 1 to 3 (as they would for
a three-dimensional problem), then this tensor could be represented as a two-dimensional
array with 9 elements.

\medskip


\noindent\textbf{Basis Vectors}


\noindent\textbf{Formal Definition}

\subsubsection{Transformation Rules}

\subsubsection{Allowed Operations}


\section{Vectors and Matrices Defined as Tensors}\label{Sec:VecMatAsTensors}

\subsection{Vectors}
A vector is a first-order tensor. It is a tensor of rank 1.

\subsubsection{Tensor Definition of Vectors}

\subsubsection{Allowed Operations as Tensors}

\subsection{Matrices}
A matrix is a second-order tensor. It is a tensor of rank 2.

\subsubsection{Tensor Definition of Matrices}

\subsubsection{Allowed Operations as Matrices}


\section{Requirements}\label{Sec:Req}
This section contains requirements for the system, categorized into
\textit{general requirements} which are requirements the system must
follow, and \textit{test-driven requirements} which are example scientific
problems the system must be able to generate proper documentation and 
code to solve.

\subsection{General Requirements}

\begin{enumerate}[label={R\arabic*.}]
\item The system shall contain an internal representation of tensors.
\item The system shall contain a smart constructor for vectors, represented
      internally as rank-1 tensors.
\item The system shall contain a smart constructor for matrices, represented
      internally as rank-2 tensors.
\item The system shall allow the specification of tensor, vector, and
      matrix operations with specification-time fixed and variable sizes.
\item The system shall support at least the vector, matrix, and tensor operations
      defined in Section~\ref{Sec:MathematicalDefinitions} of this document.
\item The system shall allow the generation of documentation using Einstein
      notation for vector, matrix, and tensors.
\item The system shall allow the generation of code to perform the operations
      for vectors, matrices, and tensors.
\item The system shall ensure or check the validity of operations at specification 
      time, generation-time, or runtime, as appropriate.
\item The addition of these new features to Drasil shall not break any existing
      examples.
\item The system shall support the generation of artifacts to solve the scientific
      computing problems listed in Section~\ref{Subsec:TDR} of this document.
\end{enumerate}

\subsection{Test-Driven Requirements}\label{Subsec:TDR}
This section describes several example uses of tensors, vectors,
and matrices which will be used to test the system. Since these problems
have oracles, both in the form of the required generated documentation as well as
computationally in terms of the results of the generated code, they are considered
to be ``test-driven'' requirements of the system. That is, the system must be
able to generate proper artifacts that are satisfactory to these examples. 

\subsubsection{Linear Operations}
TODO: Basic transformations using transformation matrices (scaling, rotation, etc.)

\subsubsection{Slope Stability Problems}

\subsubsection{Stress-Strain Problems}








\newpage

\bibliographystyle {plainnat}
\bibliography {../../refs/References}

\end{document}